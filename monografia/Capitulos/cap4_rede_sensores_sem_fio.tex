\chapter{Rede de Sensores Sem Fio}
\label{cap:4}

\section{Considerações Iniciais}
A integração de sensores em estruturas, máquinas e ambientes, associada com a entrega eficiente das
informações observadas, pode oferecer inúmeros benefícios para a sociedade, como prevenção de catástrofes,
conservação de recursos naturais e aprimoramento de comodidade e segurança. Isso pode ser alcançado através
da implantação de uma rede de sensores sem fio (RSSF) \cite{townsend_arms2005}.

Uma RSSF pode ser definida como uma rede de dispositivos, denominados nós, que podem monitorar o ambiente e
comunicar a informação adquirida através de ligações sem fio. Esses dados são transmitidos, diretamente ou por
múltiplos saltos, dependendo da topologia da rede, para um dispositivo principal, que pode estar conectacdo à
outras redes, como a \textit{Internet}, e que oferece uma interface de interação entre o usuário e a RSSF
\cite{buratti2011}.

Segundo \citeonline{hai_nayak_stojmenovic2010}, há três tipos de aplicação para RSSFs:
\begin{itemize}
	\item \textbf{Orientada à Eventos:} os sensores reportam a informação obtida ao controlador quando alguma
	evento específico é detectado;
	\item \textbf{Periódica:} os sensores adquirem informações sobre o ambiente em tempos determinados e as enviam
	para o controlador periodicamente;
	\item \textbf{Sob Demanda:} os usuários é quem decidem quando obter dados ao enviar uma solicitação à RSSF e
	aguardar respectiva resposta.
\end{itemize}

Quanto à sua aplicação em um sistema de automação residencial, de acordo com os princípios mencionados no
Capítulo \ref{cap:2}, ela torna-se parte integrante da rede de controle, pois atua apenas sobre os sensores e
atuadores e não sobre os aparelhos de consumo e possibilita com que o residente a gerencie através de um
dispositivo de controle remoto.

Para \cite{townsend_arms2005}, a RSSF ideal deve ser escalável, consumir muito pouca energia, ter capacidade
de rápida aquisição de dados, ser confiável e precisa a longo prazo, possuir baixo custo de desenvolvimento e
instalação e não necessitar manutenção significativa.

Em relação à sua implementação, alguns aspectos importantes devem ser definidos antes do processo de
desenvolvimento, sendo eles a composição dos nós, a topologia da rede e os mecanismos de segurança.

\section{Composição dos Nós}
Os nós de uma RSSF são compostos de cinco componentes principais, sende eles uma unidade controladora, um
dispositivo de armazenamento de memória, sensores e atuadores, um transceptor\footnote{Transceptor refere-se a
um dispositivo que atua como transmissor e receptor.} sem fio e uma fonte de energia. Cada um desses
componentes deve operar seguindo um equilíbrio entre o menor consumo de energia possível e a necessidade de
cumprir suas tarefas \cite{karl_willig2005}.

Na prática, a unidade controladora e o armazenamento de memória tornam-se um só componente com o uso de
microcontroladores, ou MCU (\textit{Microcontroller Unit}), para cumprir tais funções. Desse modo, a
composição de um nó de uma RSSF é normalmente conforme ilustrado na figura \ref{figura:node}.

\begin{figure}[h]
	\caption{Composição de um nó em uma RSSF}
	\centering
	\includegraphics[scale=0.5]{../images/node.png}
	\hspace{\linewidth}
	Fonte: elaborada pelo autor
	\label{figura:node}
\end{figure}

\subsection{Microcontrolador}
Microcontroladores são geralmente definidos como computadores completos em um único \textit{chip}, pois
consistem de um núcleo de processamento com memórias de dados e de programa, pinos de E/S configuráveis e
outras funcionalidades, dependendo do modelo utilizado, como temporizadores, portas de comunicação serial,
conversor analógico-digital, etc \cite{williams2014,kuorilehto2007}.

Além disso, algumas características como flexibilidade em conectar dispositivos externos, conjunto de
instruções favoráveis à processamento de tempo-crítico e baixo consumo elétrico fazem com que os MCUs sejam
amplamente utilizados em diversas aplicações, como sistemas embarcados e, evidentemente, redes de sensores sem
fio \cite{karl_willig2005}.

Atualmente, existem inúmeros modelos de MCU de fabricantes diferentes no mercado, e um dos mais populares são
os modelos \texttt{AVR} da empresa \textit{Atmel Corporation}. O principal motivo pela difusão desses modelos,
além do baixo preço, é que eles oferecem \textit{softwares} abertos e gratuitos para realizar a implementação
do código embarcado. Um deles é o compilador \texttt{avr-gcc}, que é uma montagem do \textit{GNU Compiler Collection}
específica para os AVR e utiliza a biblioteca \texttt{AVR-Libc}, que fornece um subconjunto da biblioteca C
padrão. Além disso, há também o programa \texttt{avrdude}, que é o responsável em transferir o código binário
gerado para o microcontrolador.

Conforme mostra a figura \ref{figura:avr}, os microcontroladores AVR estão disponíveis em diversos tamanhos e
pacotes, sendo que as funcionalidades embutidas, quantidade de memória disponível e outros atributos dependem
do modelo utilizado.

\begin{figure}[h]
	\caption{Microcontroladores AVR}
	\centering
	\includegraphics[scale=0.5]{../images/avr.jpg}
	\hspace{\linewidth}
	Fonte: http://home.roboticlab.eu
	\label{figura:avr}
\end{figure}

Os modelos AVR são os mesmos utilizados pelo Arduino, que nada mais é que uma camada de abstração ao MCU e que
oferece algumas facilidades como conectores mais convenientes, bibliotecas implementadas, entre outras.
Contudo, o objetivo principal do Arduino é permitir que novatos e pessoas fora da área possam realizar
prototipações de computação de baixo nível. Embora seja uma ótima plataforma, seu custo chega a ser quatro vez
mais que um AVR separado \cite{trevennor2012}.

\subsection{Sensor/Atuador}
Ver Capítulo \ref{cap:3}.

\subsection{Transceptor}
O uso da comunicação por radiofrequcência (RF) tem se tornado cada vez mais abrangente, indo desde as
aplicações tradicionais, como transmissão de sinais de rádio e televisão, para as mais diversas utilidades,
como monitoramento de pacientes em um hospital, \textit{mouses} e teclados sem fio, identificação por
radiofrequência (RFID) e, naturalmente, redes de sensores sem fio \cite{misra2001}.

Embora transceptores baseados em ondas ópticas possuam uma eficiência energética melhor, sua necessidade por
uma condição de linha de visão devido ao seu comportamento direcional faz com que a comunicação baseada em RF
seja a mais relevante para a construção de uma RSSF, pois, além de ser omnidirecional, provê uma distância de
alcance e uma taxa de tranferência de dados relativamente grandes \cite{kuorilehto2007,karl_willig2005}.

Conforme mencionado no Capítulo \ref{cap:1}, os módulos transceptores RF que implementam protocolos de
comunicação avançados como o ZigBee e o Wi-Fi deixam a desejar em relação ao preço e consumo energético,
sendo necessário optar por módulos mais simples.

Uma possibilidade é utilizar o transceptor \texttt{CC2500} da empresa \textit{Texas Instruments}, que
implementa o padrão IEEE 802.15.4. Devido ao baixo consumo elétrico e ótimo custo-benefício, esse dispositivo
é amplamente utilizado.

Outro transceptor de rádio frequência bastante difundido é o \texttt{nRF24L01+} da empresa \textit{Nordic
Semiconductor}. Embora possui a desvantagem de não seguir o padrão aberto da IEEE, esse dispositivo apresenta
diversas vantagens em relação aos demais módulos dessa categoria, podendo ser observadas no quadro
\ref{quadro:transceivers}, que exibe uma comparação entre os dois transceptores mencionados e dois que
implementam protocolos de comunicação mais avançados. As informações foram retiradas de seus respectivos
\textit{datasheets} e os preços são de módulos prontos para o uso encontrados no mercado.

\begin{table}[H]
	\centering
	\resizebox{\textwidth}{!} {
		\begin{tabular}{|l|c|c|c|c|c|c|}
		\hline
		Transceptor   & Padrão/Protocolo       & Taxa de Transmissão Máxima & Consumo RX & Consumo TX &
		Consumo Espera & Preço     \\ \hline \hline
		nRF24L01+     & Enhanced ShockBurst    & 2 Mbps   & \SI{13.5}{\milli \ampere} & \SI{11.3}{\milli
		\ampere} & \SI{26}{\micro \ampere} & U\$ 3,00  \\ \hline
		CC2500        & IEEE 802.15.4          & 500 Kbps & \SI{17}{\milli \ampere}   & \SI{21.2} {\milli
		\ampere}  & \SI{1.5}{\milli \ampere} & U\$ 4,00  \\ \hline
		xBee Series 1 & IEEE 802.15.4 / ZigBee & 250 Kbps & \SI{50}{\milli \ampere}   & \SI{45}{\milli
		\ampere} & \SI{10}{\micro \ampere}  & U\$ 25,00 \\ \hline
		ESP8266       & Wi-Fi                  & 54 Mbps  & \SI{60}{\milli \ampere}   & \SI{145}{\milli
		\ampere} & \SI{0.9}{\milli \ampere}& U\$ 7,00  \\ \hline
		\end{tabular}
	}
	\caption{Comparação entre módulos transceptores}
	\label{quadro:transceivers}
\end{table}

O \texttt{nRF24L01+}, ilustrado na figura \ref{figura:nrf}, ainda oferece serviços como reconhecimento e
retransmissão de pacotes automáticos, diminuindo o número de comunicação com a unidade microtronladora tal
como o processamento utilizado pela mesma. Dessa forma, além de reduzir ainda mais o consumo elétrico
necessário, possibilita uma implementação eficiente utilizando microcontroladores simples e baratos.

\begin{figure}[h]
	\caption{Módulo Transceptor NRF24L01+}
	\centering
	\includegraphics[scale=0.25]{../images/nrf.jpg}
	\hspace{\linewidth}
	Fonte: http://www.techmake.com
	\label{figura:nrf}
\end{figure}

\subsection{Fonte de Energia}
Consiste de um mecanismo de armazenamento de energia (normalmente uma bateria não recarregável), um regulador
de tensão e, opcionalmente, uma unidade de obtenção energética. Essa obtenção pode ser através de diversos
fenômenos, como a luz, vibrações, variação de temperatura, variação de pressão, entre oturos
\cite{kuorilehto2007,karl_willig2005}.

A característica de energia excassa que aplicações RSSF possuem, implica em alguns requerimentos de eficiência
energética no planejamento dos nós, ou seja, o consumo deve ser minimizado. Há algumas estratágias que podem
ser usadas para atingir esse objetivo, como reduzir a quantidade de informações transmitidas através de
compreensão de dados ou redução, ativar o sensor apenas quando estiver coletando dados, implementar uma
estratégia orientada à eventos para a coleta e transmissão, etc \cite{kuorilehto2007,townsend_arms2005}.

\section{Topologia da Rede}
\subsection{Estrela}
Em uma rede deste tipo, cada nó se comunica diretamente com a estação base (controlador). A vantagem
desta topologia consiste na sua simplicidade e na capacidade de manter o menor consumo de energia possível,
entretanto, o alcance fica limitado ao alcance do módulo transceptor, dificultando a escalabilidade da solução
\cite{dargie_poellabauer2010,townsend_arms2005}.

\begin{figure}[h]
	\caption{Rede Estrela}
	\centering
	\includegraphics[scale=0.25]{../images/star.png}
	\hspace{\linewidth}
	Fonte: https://www.wikipedia.org/
	\label{figura:star}
\end{figure}

\subsection{Malha}
Nessa rede, cada nó pode se comunicar diretamente com qualquer outro nó dentro de seu alcance, o que permite a
chamada comunicação \textit{multihop}, ou seja, caso um nó queira se comunicar com outro fora de seu alcance,
ele pode utilizar um intermediário para transmitir a mensagem adiante. Dessa forma, torna mais fácil escalar a
aplicação e diminui a necessidade de uma estação central para certas operações. Em contrapartida, a
complexidade aumenta pois surge a necessidade de criar mecanismos de roteamento, além de aumentar o consumo de
energia elétrica \cite{townsend_arms2005}.

\begin{figure}[h]
	\caption{Rede Malha}
	\centering
	\includegraphics[scale=0.3]{../images/mesh.png}
	\hspace{\linewidth}
	Fonte: https://www.wikipedia.org/
	\label{figura:mesh}
\end{figure}

\subsection{Árvore}
Também chamada de hierárquica, essa topologia permite uma comunicação \textit{multihop}, porém com um protocolo de
roteamento mais simples, pois nela existe apenas uma rota para cada nó remoto a partir do controlador. Sendo
assim, consome menos energia e é mais simples de implementar que uma rede malha e possui uma área de
abrangência maior que uma rede estrela.

\begin{figure}[h]
	\caption{Rede Árvore}
	\centering
	\includegraphics[scale=0.25]{../images/tree.png}
	\hspace{\linewidth}
	Fonte: https://www.wikipedia.org/
	\label{figura:tree}
\end{figure}
