\chapter {Automação Residencial}
\label{cap:1}

\section {Considerações Iniciais}
O ser humano desde o início da sua existência, usou de artifícios para auxiliá-lo no seu dia-a-dia. Esses
artifícios podiam ser desde ferramentas para auxílio na agricultura como a enxada, ou animais como o cavalo
para transporte. Esses artifícios requeriam ao usuário algo que chamamos de interfaces naturais, ou seja
usavam de seus sentidos para o manuseio dessas ferramentas. Com a substituição do trabalho braçal pelas
máquinas, surgem então as interfaces de máquina, onde usuário tinha que manusear alavancas, botões, lâmpadas e
afins \cite{kirner2007}.

Com o advento do computador, sendo o ENIAC o primeiro computador digital eletrônico, cria-se a interface
humano-computador, que revolucionou a forma como o homem interage com as ferramentas de trabalho, exigindo um
conhecimento abstrato, pois o conhecimento do que é real já não era mais suficiente \cite{eniac,kirner2007}.

A evolução dessas formas de interação, trouxe inúmeros benefícios como: uma maior agilidade em toda a produção
nas industrias, a diminuição de falhas, entre muitos outros. Porém trouxe também a necessidade de treinamento
e adaptação por parte dos usuários, por causa da sofisticação dessas interfaces. Contudo os desenvolvedores
sempre buscaram criar interfaces amigáveis, visando facilitar cada vez mais o uso do computador
\cite{kirner2007}.

A grande evolução do \textit{hardware} e do \textit{software} proporcionou o desenvolvimento de formas para
que o computador se adapta-se ao ser humano. Tecnologias como o \textit{mouse} ou a mais recente:
\textit{touchscreen}, proporcionam uma interação muito mais fluida, com o usuário. Olhando para o
\textit{software} podemos citar inicialmente as interfaces baseadas em comandos a exemplo do \verb'DOS'
(\textit{Disk Operating System}), um pouco mais a frente temos as interfaces baseadas em janelas, nesse ponto
da história é inserido o \textit{mouse}, trazendo uma interação maior do usuário com o sistema, e mais
recentemente as interfaces baseadas no toque (\textit{touchscreen}), introduzindo também os
\textit{smartphones} \cite{kirner2007}.

Nesse contexto que entra a \verb'RV' (Realidade Virtual) sendo uma inovadora forma de interface
usuário-máquina, pois a imersão do usuário no mundo virtual e a possibilidade da comunicação do mesmo com esse
mundo, abre um leque de possibilidades de uso dessa tecnologia.

Toda essa evolução tecnológica propiciou nos anos 90 a criação da \verb'RA' (Realidade Aumentada), mesclando
os modelos tridimensionais da \verb'RV' com o mundo como conhecemos, fazendo com que para a visão do usuário
esses dois mundos coexistam.

Representações da realidade ou até mesmo da imaginação e formas de como interagir com o mundo a sua volta, são
necessidades bastante recorrentes na história da humanidade. As mais variadas formas e ferramentas foram
desenvolvidas para suprir essas necessidades. Essas necessidades são supridas pelas tecnicas de \verb'RA' e
\verb'RV' e serão melhor abordadas nos próximos parágrafos. \cite{kirner2007,kirner2006}.

\section {Realidade Virtual}
O ser humano em toda sua história buscou representar os eventos de seu cotidiano e as vezes até acontecimentos
de sua imaginação das mais variadas maneiras. Essa necessidade criou muitas formas de expressão, tais como:
desenhos primitivos em paredes, pinturas e figuras, música, dança, opera, teatro, cinema, entre outras
expressões artísticas. O desenvolvimento do computador potencializou essas expressões, viabilizando a criação
do que conhecemos como multimídia que tem como elementos: textos, sons, vídeos, imagens, animações e mais
recentemente a hipermidia que nos permite interagir de uma forma não linear entre esses diversos
elementos \cite{kirner2006}.

Com a evolução da tecnologia e dos componentes de hardware, soluções computacionais cada vez mais próximas à
nossa realidade foram desenvolvidas, convergindo no que temos atualmente, como ambientes tridimensionais
passíveis de interação com o usuário, que podem ser chamados de ambientes de Realidade Virtual
\cite{kirner2006}.

A tendência ao criar aplicações usando \verb'RV' é de representar o mundo real, porém simular o imaginário se
faz de grande importância também, por exemplo quando se faz necessário representar uma ideia para poder ser
analisada antes de por em prática, se essa ideia puder ser representada em uma forma visual, usando elementos
tridimensionais, ambientes virtuais, a chance de que algo substancial seja desenvolvido é maior do que se
fosse exposto em diagramas, ou imagens no papel.

Há algum tempo atrás representar o imaginário poderia ser feito somente de forma verbal, escrita ou  em
desenhos, esculturas, maquetes. Dessa forma, levava-se muito mais tempo, custo e na maioria das vezes não era
possível representar de forma coerente o que se necessitava \cite{kirner2006}.

Com a criação da \verb'RV', o processo de simular o real e representar o imaginário tornou-se mais simples e
mais fácil de ser feito. A \verb'RV' trouxe a possibilidade de uma maior interação entre a representação e o
usuário, quebrando algumas barreiras existentes até então como a tela do computador, permitindo a atuação do
usuário no espaço tridimensional \cite{kirner2006}.

Além de ser uma forma mais palpável de vermos o real ou o imaginário, aplicações feitas com \verb'RV' podem
potencializar os sentidos, pois como é um ambiente virtual moldado a se parecer com o ambiente real, nada
impede de alterar a capacidade humana: ver mais longe, mais perto, correr mais rápido ou mais lento, ouvir
mais alto ou mais baixo, ou até mesmo viajar grandes distâncias em frações de segundo. As possibilidades são
infinitas ampliando também a gama de aplicações que podem ser feitas com a \verb'RV' \cite{kirner2006}.

Apenas moldar um ambiente virtual baseado em um ambiente real não é o suficiente para que haja uma experiência
gratificante ao usuário. É importante também possibilitar a interação com esse ambiente, para isso são usados
os recursos de programação, permitindo com que o usuário possa atuar nesse ambiente virtual, podendo interagir
ou alterar algo nesse cenário \cite{kirner2006}.

Dada toda essa constextualização vamos a algumas definições de Realidade Virtual:
\begin{enumerate}
\item É a representação de um ambiente real em um ambiente virtual dando a possibilidade de interação com o usuário.
\item É uma “interface avançada do usuário” para acessar aplicações executadas no computador, propiciando a visualização, movimentação e interação do usuário, em tempo real, em ambientes tridimensionais gerados por computador \cite{kirner2006,kirner2007}.
\item Permite ao usuário visualizar ambientes tridimensionais, podendo se movimentar dentro deles e interagir com os elementos que estão na cena, elementos esses que podem ter comportamentos diferentes, conforme a interação do usuário \cite{kirner2006}.
\end{enumerate}

A interação do usuário com o ambiente tridimensional é algo de extrema importância em uma aplicação de \verb'RV', esta realacionada com a capacidade que o computador possui de analisar e identificar as ações do usuário, verificando se essas ações irão ou não modificar algo na cena. O usuário interagindo com os elementos do ambiente tridimensional, podendo ver suas ações modificarem aspectos da cena, torna a experiência muito mais completa e natural, pois um dos principais objetos da \verb'RV' é tornar o mundo virtual o mais próximo do que vemos no mundo real \cite{kirner2006}.

Para que essa interação flua de uma forma mais agradável, se faz necessário o uso de algum aparato tecnológico como um mouse, ou uma luva, ou algum outro dispositivo de apoio. É necessário também que o ambiente computacional que a aplicação esteja sendo executada, possua um hardware compatível e adequado, apresentando um bom desempenho ao renderizar modelos tridimensionais em tempo real, e dar suporte para a utilização desses dispositivos de apoio \cite{kirner2006}.

A ação mais simples que o usuário consegue fazer nos ambientes virtuais é caminhar pela cena, porém, usando apenas esse artifício nada é alterado na cena. Com o uso dos dispositivos de apoio, as possibilidades aumentam, podemos tocar em objetos e manuseá-los, alterar e movimentar objetos nas cenas, iteragir com personagens e assim por diante, fazendo com que a experiência seja muito mais fluida.

Apesar da experiência de poder iteragir com um ambiente tridimensional, através de dispositivos como luvas, capacete, óculos que simulam o movimento do usuário no ambiente virtual, a necessidade de usar algum dispositivo acoplado em alguma parte do corpo nem sempre é satisfatória. Pensando nisso foi desenvolvido os dispositivos de captura de movimento, fazendo com o usuário não precise usar algum acessório, pois seus movimentos serão captados por uma câmera simulando-os no ambiente virtual, um exemplo de dispositivo que faz isso é o \textit{Kinect} desenvolvido pela Microsoft para o console XBOX \cite{kinect}.

\subsection {Tipos de Sistemas de Realidade Virtual}
A \verb'RV' pode ser classificada conforme o nível de imersão do usuário no ambiente virtualizado, podendo ser imersiva ou não imersiva \cite{kirner2006}.

A  \verb'RV' imersiva acontece quando o usuário é ''inserido'' inteiramente no mundo virtual, usando dispositivios sensoriais que capturam os seus sentidos: movimento da cabeça, movimento das mãos, passos. Esses dispositivos visam proporsionar a sensação de estar dentro do ambiente virtual, podendo interagir com esse mundo de uma forma mais completa \cite{kirner2006}.

Já a \verb'RV' não imersiva é caracterizada por não haver a total interação do usuário com o ambiente virtualizado, limitando a visão do usuário sobre o mundo virtual através de um monitor ou uma projeção da imagem. Apesar da imersão não ser total, o usuário tem a sensação de estar predominantemente no mundo virtual \cite{kirner2006}.


\subsection{Sistemas de Realidade Virtual}
Para classificar qualitativamente um sistema de realidade virtual, deve se considerar quatro elementos, o ambiente virtual, o ambiente computacional, a interação com o usuário e a tecnologia de realidade virtual \cite{kirner2006}.

\subsubsection{Ambiente Virtual}
O ambiente virtual aborda questões como: construção do modelo tridimensional, características dinâmicas do ambiente, características de iluminação e detecção de colisão \cite{kirner2006}.

O ambiente virtual pode ter diversas formas que representam algo no mundo real como: prédios, casas, estradas, ruas entre outros. Por isso a precisão geométrica, as cores, texturas e iluminação são de grande importância nas cenas. Contudo, podem existir casos em que a representação do ambiente virtual, não seja baseada no mundo real, construindo o que chamamos de ambiente abstrato, não retirando a importância de se preocupar com as cores, texturas e iluminação para uma melhor visualização e imersão. A casos também em que o ambiente virtual é utilizado para fazer uma simulação física, na qual, a precisão do comportamento em questão é muito mais importante do que a forma como é representado visualmente \cite{kirner2006}.

Todo ambiente virtual contém objetos virtuais tridimensionais. Esses objetos possuem cor, textura, características dinâmicas, restrições físicas e sons. Esses objetos são representações de objetos do mundo real e são classificados em \verb'RV' como: estáticos e dinâmicos. Um objeto estático e aquele que há como movê-lo na cena, já o objeto dinâmico podemos retirá-lo de sua posição original \cite{kirner2006}.

Os objetos virtuais possuem características físicas: peso, dimensões que afetam na maneira como esse objeto poderá ser movimentado na cena. Há também as restrições físicas

\subsubsection{Ambiente Computacional}

\subsubsection{Interação com o usuário}

\subsubsection{Tecnologia de Realidade Virtual}

\section {Realidade Aumentada}

Com o a criação da \verb'RV', a necessidade de demostrar a realidade em um ambiente virtual está bem representada. Seguindo desse conceito e somando todo o avanço tecnológico que a computação vem apresentando e as melhorias em multimídia, surge a possibilidade da junção entre o mundo real e o mundo virtual. Da realidade virtual pegamos os modelos tridimensionais, sendo eles imagens ou até mesmo textos. Da multimídia usamos a captura de imagens através das câmeras e os dispositivos sensoriais que possibilitam a interação com o usuário. Juntando essas ferramentas, e atrelando a elas todo o poder computacional que dispomos na atualidade, cria-se o conceito de \verb'RA' (Realidade Aumentada).

A \verb'RA' nada mais é que a mistura do mundo virtual e do mundo real, fazendo com que para o usuário seja um mundo só. A \verb'RA' tem o intuito de enriquecer o mundo real, com objetos do mundo virtual. Esse enriquecimento, nos trás uma nova forma de interagir com o mundo real, gerando as mais variadas aplicações dessa tecnologia.




\section {Considerações Finais}
