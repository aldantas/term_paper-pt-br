\chapter {Automação Residencial}
\label{cap:1}

\section {Considerações Iniciais}
De acordo com \citeonline{groover1987}, automação é a tecnologia pela qual um processo é realizado
sem assistência humana através de um sistema de controle, instruções de programa e alguma forma de energia,
sendo a elétrica a mais comum.

Seguindo esta mesma definição, uma automação residencial é um produto ou serviço que proporciona algum nível de
ação ou mensagem para o ambiente domiciliar, um evento que foi gerado sem a intervenção direta do morador. Um
despertador ou um alarme de incêndio são exemplos disso, porém, esses dispositivos autônomos não
necessariamente possuem um mecanismo de comunicação entre eles, limitando o nível da automação e inteligência
da solução. \cite{riley2012}

A automação residencial é um componente fundamental para que se possa construir casas inteligentes, ou seja,
ambientes inteligentes que interagem dinamicamente e respondem prontamente às necessidade dos ocupantes a às
mudanças condicionais de uma maneira adaptativa. \cite{al-qutayri2010}

A variade de aplicações suportadas por uma casa inteligente é bastante abrangente. Algumas das mais comuns incluem
monitoramento e controle do ambiente, segurança, entretenimento, serviços baseados em localização, cuidados
de crianças e idodos, entre outros. \cite{al-qutayri2010}

É desejável que essa automação seja pervasiva, ou seja, que a interação entre o usuário e o ambiente ocorra
naturalmente. Os dispositivos de uma solução pervasiva, ou ubíqua, possuem três características fundamentais:
miniaturalização, comunicação e autonomia. \cite{lalanda2010}

O atendimento desses critérios está ocorrendo de uma forma crescente nos últimos anos através da evolução dos
equipamentos computacionais, tornando ainda mais acessível o desenvolvimento de uma automação residencial,
seja por empresas ou entusiastas na área.

\section{Conceitos Chaves}
Para \citeonline{kyas2013}, uma automação residencial consiste em cinco blocos de construção:

\begin{itemize}
	\item Dispotivos sob controle;
	\item Dispositivos de controle remoto;
	\item Rede de controle;
	\item Controlador;
	\item Sensores e atuadores.
\end{itemize}

\subsection{Dispositivos Sob Controle}
Trata-se dos eletrodomésticos pertencentes à uma residência, como geladeira, televisão, etc. Atualmente,
esses dispositivos já possuem diversas funcionalidades embutidas, podendo ser chamados de inteligentes por si
só.

Um grande exemplo são as \textit{smtartTVs}, que possibilitam que possibilitam com que a televisão deixe de
ser apenas um aparelho de reprodução de conteúdo par ser um centro de entretenimento interativo.




\section{Desafios}

\section {Considerações Finais}
