\chapter{Sensores e Atuadores}
\label{cap:3}

\section{Considerações Iniciais}
Um sensor é um dispositivo que detecta ou mede um quantitativo físico e o transforma, normalmente, em um sinal
elétrico. O dispositivo oposto é o atuador, que converte um sinal elétrico para alguma ação, normalmente
mecânica \cite{sinclair2001}.

Um transdutor é definido como qualquer dispositivo que converte um tipo de energia para outro. Alguns
autores diferenciam-os de sensores, seja alegando que transdutores possuem uma preocupação maior com a eficiência
da conversão de energia ou que o sensor é apenas uma parte do transdutor responsável por detectar a variável
do ambiente \cite{sinclair2001,kondrasovas2013}.

Contudo, neste trabalho considera-se que, por definição, sensores e atuadores são tipos de transdutores.

\section{Sensores}
Avanços tecnológicos recentes tem permitido o desenvolvimento de dispositivos sensores de baixo custo, baixo
consumo de energia e pequenos portes. Eles podem medir distância, direção, velocidade, umidade, temperatura, luz,
vibração, pressão, propriedades acústicas e muitos outros atributos \cite{hai_nayak_stojmenovic2010}.

De acordo com \citeonline{karl_willig2005}, os sensores podem ser divididos em três categorias:
\begin{itemize}
	\item \textbf{Passivos e omnidirecionais:} medem informações físicas sem manipular o ambiente,
	utilizando apenas os fenômenos existentes (vibração, luz, radiação, etc). Além disso, não possuem
	noção de direção envolvida na medição. A maioria dos sensores pertencem à esta categoria, alguns
	exemplos são termômetros, sensores de luz, umidade, fumaça, entre outros.
	\item \textbf{Passivos e de feixe estreito:} diferenciam-se dos anteriores pois possuem uma noção bem
	definida de direção. Um exemplo típico é uma câmera.
	\item \textbf{Ativos:} ao contrário dos anteriores, sensores deste tipo emitem algum tipo de sinal,
	como ondas ou elétrons, e captam as informações através do reflexo desses sinais emitidos. Os
	exemplos mais conhecidos são os sensores sonares, radares e sísmicos.
\end{itemize}

Também podem ser classificados quanto à referência selecionada, sendo ela absoluta ou relativa. Um sensor
absoluto converte um estímulo para uma escala física absoluta que é independente das condições de medição, já
um sensor relativo produz um sinal que se refere à algum caso especial. Os sensores de pressão, por exemplo,
podem ser absolutos, cujo sinal produzido é relativo ao vácuo (zero absoluto na escala de pressão), ou
relativos, que poduzem sinais referentes à algum patamar, como a pressão atmosférica \cite{fraden2010}.

Uma maneira lógica de classificá-los é através da propriedade física que está sendo medida, como
temperatura, pressão, movimento, etc \cite{kenny_walt2005}.

Podem também ser divididos em analógicos e digitais. Segundo \citeonline{thomazini_albuquerque2005}, um sensor
analógico é capaz de assumir qualquer valor em seu sinal de saída ao longo do tempo, desde que esteja dentro
da sua faixa de operação, já um digital pode assumir apenas dois valores, que podem ser interpretados como
zero ou um, após serem convertidos pelo circuito eletrônico do transdutor.

Há diversas características relevantes quando se trata de sensores, uma delas é a resolução, que mede a
menor alteração do quantitativo físico que um sensor consegue detectar \cite{sinclair2001}.

Outra característica importante é a área de cobertura ou alcance de um sensor, que consiste na distância em
que ele consegue captar a informação de maneira precisa e confiável \cite{karl_willig2005}.

Além desses, existem alguns outros atributos que caracterizam um sensor, sendo eles: sensibilidade, a relação
entre o sinal físico de entrada e o elétrico de saída; exatidão, a diferença entre o valor real e o valor
gerado pelo sensor; precisão, a capacidade de reproduzir os resultados que foram obtidos experimentalmente da
mesma forma; ruído, produzido pelo sensor a adicionado ao sinal de saída; largura de banda, velocidade com que
o sensor consegue prover uma corrente de leitura \cite{kenny_walt2005,kondrasovas2013}.

\subsection{Temperatura}
Devido à grande significância de seu efeito em materiais, a temperatura é a variável mais medida e especifica
o grau de ``quentura'' ou ``frieza'' referenciada à uma escala específica.\cite{fontes2005}.

Calor é a quantidade total de energia cinética das moléculas e átomos em um objeto ou sistema especifico,
sendo assim, a temperatura de um corpo é proporcional à taxa de energia cinética das moléculas
\cite{peeters_peetermans_indesteege2007}.
