\chapter{Conclusões e Trabalhos Futuros}
\label{cap:conclusao}
Este trabalho obteve como resultado uma implementação de RSSF para ambientes não críticos com uma melhor
eficiência energética e um custo monetário significativamente menor do que as soluções normalmente encontradas
no mercado.

Isso possibilita com que mais pessoas tenham acesso a esse tipo de tecnologia, permitindo-as usufruir das
vantagens que um sistema de automação traz ao cotidiano.

Embora a transmissão dos dados tenha sido bem sucedida, para que a implementação seja utilizada de modo
funcional é necessário ainda desenvolver a parte do sistema computacional, que é responsável por gerenciar a
RSSF, integrá-la ao sistema de automação residencial, caso houver, e fornecer aos usuários um modo de
interação com os transdutores.

Uma outra possibilidade de trabalho futuro é analisar a viabilidade em aplicar técnicas de segurança, como
criptografia, na comunicação. Isso permitiria extender a usabilidade da RSSF, porém, deve ser levado em
consideração o baixo poder de processamento disponível.

É possível ainda estudar a possibilidade em integrar a RSSF com fontes de energias renováveis e com sistemas
de \textit{Smart Grid}, possibilitando uma melhor gerência de recursos e uma maior segurança aos eletrônicos
de consumo.

O processo de pesquisa e desenvolvimento fez com que eu me familiariza-se ainda mais com conceitos de
automação e sistemas embarcados, áreas que eu particularmente me interesso. Portanto, concluo que pessoalmente
o resultado foi satisfatório.
