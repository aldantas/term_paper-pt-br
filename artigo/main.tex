\documentclass[a4,11pt]{report}

\usepackage[brazil]{babel}      % para texto em Português
%\usepackage[english]{babel}    % para texto em Inglês

%\usepackage[latin1]{inputenc}   % para acentuação em Português OU
\usepackage[utf8]{inputenc}   % para acentuação em Português com o uso do Unicode,
% mude a codificação desse template para utf-8

%%
%% POR FAVOR, NÃO FAÇA MUDANÇAS NESSE TEMPLATE QUE ACARRETEM  EM
%% ALTERAÇÃO NA FORMATAÇÃO FINAL DO TEXTO
%%
\usepackage{graphics}
\usepackage{subfigure}
\usepackage{graphicx}
\usepackage{epsfig}
\usepackage[centertags]{amsmath}
\usepackage{graphicx,indentfirst,amsmath,amsfonts,amssymb,amsthm,newlfont}
\usepackage{longtable}
\usepackage{cite}
\usepackage[]{siunitx}
\usepackage[usenames,dvipsnames]{color}


\begin{document}

%********************************************************
\title{Desenvolvimento de uma Rede de Sensores Sem Fio de Baixo Custo}

\author{
 {\large  Augusto Lopez Dantas}\thanks{aldantas@protonmail.com}\\
 {\small Centro de Engenharias e Ciências Exatas, UNIOESTE, Foz do Iguaçu, PR} \\
 }

\criartitulo

\section{Introdução}

Uma rede de sensores sem fio (RSSF) é um conjunto de dispositivos que podem monitorar e controlar o ambiente e
comunicar a informação adquirida utilizando o ar como meio físico de transmissão.

Pode ser classificada em três categorias: orientada à eventos, onde a informação é reportada quando algum
evento específico é detectado; periódica, que possui tempos determinados para a coleta a comunicação dos
dados; sob demanda, na qual os usuários decidem quando obter os dados \cite{liu_nayak_stojmenovic}.

Para a aplicação de uma RSSF em um ambiente domiciliar ou de escritório, é desejável desenvolver uma solução
de baixo custo monetário e baixo consumo elétrico, pois o alcance e precisão necessários são reduzidos para
esses ambientes.

\section{Arquitetura Proposta}
Cada estação da RSSF é composta por uma unidade microcontroladora, um transceptor de radiofrequência, uma
fonte de energia e um sensor ou atuador.

Os microcontroladores utlizados foram modelos AVR da empresa \textit{Atmel Corporation}, pois disponibilizam
programas livres e gratuitos para a implementação do código embarcado.

Para transceptor, existem diversos dispositivos disponíveis no mercado, sendo quatro deles comparados na
Tabela \ref{quadro:transceivers}. As informações foram retiradas de seus respectivos \textit{datasheets} e os
preços são de módulos prontos para o uso encontrados no mercado.

\begin{table}[h]
	\caption{\small Comparação entre módulos transceptores}
	\resizebox{\textwidth}{!} {
		\begin{tabular}{|l|c|c|c|c|c|c|}
		\hline
		Transceptor   & Padrão/Protocolo       & Taxa de Transmissão Máxima & Consumo RX & Consumo TX &
		Consumo Espera & Preço     \\ \hline
		nRF24L01+     & Enhanced ShockBurst    & 2 Mbps   & \SI{13.5}{\milli \ampere} & \SI{11.3}{\milli
		\ampere} & \SI{26}{\micro \ampere} & U\$ 3,00  \\ \hline
		CC2500        & IEEE 802.15.4          & 500 Kbps & \SI{17}{\milli \ampere}   & \SI{21.2} {\milli
		\ampere}  & \SI{1.5}{\milli \ampere} & U\$ 4,00  \\ \hline
		xBee Series 1 & IEEE 802.15.4 / ZigBee & 250 Kbps & \SI{50}{\milli \ampere}   & \SI{45}{\milli
		\ampere} & \SI{10}{\micro \ampere}  & U\$ 25,00 \\ \hline
		ESP8266       & Wi-Fi                  & 54 Mbps  & \SI{60}{\milli \ampere}   & \SI{145}{\milli
		\ampere} & \SI{0.9}{\milli \ampere}& U\$ 7,00  \\ \hline
		\end{tabular}
	}
	\label{quadro:transceivers}
\end{table}

Baseando-se nesses dados, o modelo selecionado foi o nRF24L01+ da empresa \textit{Nordic Semiconductors} pois
é o que possui melhor custo-benefício em relação ao preço e consumo elétrico. Além disso, este módulo
transceptor possui algumas funcionalidades como confirmação e retransmissão de pacotes automáticos, múltiplos
canais de recepção e tamanho de mensagem dinâmico.

A topologia selecionada para a RSSF foi a topologia árvore, porque permite um alcance suficiente para uma área
domiciliar com um procolo de roteamento simples. Como a memória e o poder de processamento são escassos no mundo
dos microcontroladores, não é recomendável calcular as rotas antes de cada transmissão, nem armazenar os
caminhos disponíveis em cada nó.

Sendo assim, o computador responsável por gerenciar a RSSF é quem armazena a rota necessária para cada estação
remota. Como a topologia é árvore, há apenas um caminho conhecido para cada nó.

A transmissão por múltiplos saltos acontece através do envio dos endereços dos nós pertencentes à rota na
mensagem. O módulo nRF24L01+ permite até 32 bytes de mensagem e o endereço único de cada transceptor é de 5
bytes. Dessa forma, utiliza-se 25 bytes para envio de endereços e 7 bytes para os dados, possibilitando uma
comunicação de até 5 saltos.

A RSSF pode conter as três caracterı́sticas citadas anteriormente: sob demanda para os nós que possuem
atuadores; perı́odica para os sensores em geral; e orientada à eventos para sensores considerados crı́ticos,
como o de gás.

A implementação ainda possui um mecanismo de busca de caminho automático de modo que os nós recém inseridos
possam se comunicar com a estação principal sem a necessidade de configuração explı́cita de rotas.

\section{Conclusões}
Uma RSSF pode ser implantada como parte integrante de um sistema de automação residencial, sendo assim, um
desenvolvimento visando baixo custo e baixo consumo elétrico ajuda na popularização deste conceito que vem
crescendo nos últimos anos.

Este trabalho apresenta uma possível solução para este tipo de aplicação, utilizando tecnologias existentes e
de fácil acesso.

\bibliographystyle{unsrt}
\bibliography{mybibfile}

\end{document}
