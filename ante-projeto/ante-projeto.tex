%-----------------------------------------------------------------
% UNIOESTE - Ciência da Computação
% 4o. ano - Trabalho de Conclusão de Curso
% Profas. Teresinha Arnauts Hachisuca e Izaura
%-----------------------------------------------------------------
%DECLARAÇÃO DO TIPO DE DOCUMENTO, TAMANHO DA FOLHA E FONTE
%-----------------------------------------------------------------
\documentclass[
    % -- opções da classe memoir --
    12pt,               % tamanho da fonte
%   twoside,            % para impressão em verso e anverso. Oposto a oneside
    a4paper,            % tamanho do papel.
    % -- opções da classe abntex2 --
    %chapter=TITLE,     % títulos de capítulos convertidos em letras maiúsculas
    %section=TITLE,     % títulos de seções convertidos em letras maiúsculas
    %subsection=TITLE,  % títulos de subseções convertidos em letras maiúsculas
    %subsubsection=TITLE,% títulos de subsubseções convertidos em letras maiúsculas
    % -- opções do pacote babel --
    english,            % idioma adicional para hifenização
    brazil,             % o último idioma é o principal do documento
    ]{article}

%-----------------------------------------------------------------
%DEFINIÇÕES DOS PACOTES UTILIZADOS
%-----------------------------------------------------------------
\usepackage{cmap}               % Mapear caracteres especiais no PDF
\usepackage{lmodern}            % Usa a fonte Latin Modern
\usepackage[T1]{fontenc}        % Selecao de codigos de fonte.
\usepackage[utf8]{inputenc}     % Codificacao do documento (conversão automática dos acentos)
\usepackage{lastpage}           % Usado pela Ficha catalográfica
%\usepackage{indentfirst}       % Indenta o primeiro parágrafo de cada seção.
\usepackage{color}              % Controle das cores
\usepackage{graphicx}           % Inclusão de gráficos

\usepackage[brazil]{babel}      % Permite traduzir termos do LateX para português Brasil.
\usepackage{hyperref}           % Permite ativar hyperlinks
\usepackage{algorithm}          % pacote que de suporte a representação de algoritmos.

\usepackage{algorithmic}
\usepackage[pdftex]{geometry}
\usepackage{multirow}

% ---
% Pacotes de citações
% ---
\usepackage[brazilian,hyperpageref]{backref}     % Paginas com as citações na bibl
\usepackage[alf]{abntex2cite}   % Citações padrão ABNT
%\citebrackets[]

% ---
% CONFIGURAÇÕES DE PACOTES
% ---

% ---
% Configurações do pacote backref
% Usado sem a opção hyperpageref de backref
\renewcommand{\backrefpagesname}{Citado na(s) página(s):~}

% Texto padrão antes do número das páginas
\renewcommand{\backref}{}

% Define os textos da citação
\renewcommand*{\backrefalt}[4]{
    \ifcase #1 %
        Nenhuma citação no texto.%
    \or
        Citado na página #2.%
    \else
        Citado #1 vezes nas páginas #2.%
    \fi}%
% ---

%-----------------------------------------------------------------
%               INICIO DO DOCUMENTO - CAPA
%-----------------------------------------------------------------
\begin{document}


\begin{center}

    \textsc{
        \large
            \\universidade estadual do oeste do paraná
            \\unioeste - campus de foz do iguaçu
            \\centro de engenharias e ciências exatas
            \\curso de ciência da computação
            \\[1 cm]tcc - trabalho de conclusão de curso
    }
    \\
    [4 cm]
    \large Proposta de Trabalho de Conclusão de Curso
    \\
    \textbf{
        \textsc{Desenvolvimento de um protocolo de comunicação utilizando rádio frequência para uma rede de sensores}
    }
    \\[5 cm]Augusto Lopez Dantas
    \\Orientador: Jorge Habib Hanna El Khouri
    \\[2 cm]Foz do Iguaçu, 04 de maio de 2015
    \cite{monard2003}

\end{center}

\thispagestyle{empty}

%-----------------------------------------------------------------
%               IDENTIFICAÇÃO DO PROJETO
%-----------------------------------------------------------------
\section{Identificação}

    \subsection{Área e Linha de Pesquisa}
        \noindent Grande Área: Ciências Exatas e da Terra
        \\Código: 10000003
        \\[1 cm]Linha de Pesquisa: Ciência da Computação
        \\Código: 10300007
        \\[1 cm]Especialidade: \{nome da especialidade\}
        \\Código: \{código da especialidade\}

    \subsection{Palavras-chave}

        \begin{enumerate}
            \item Domótica
            \item Comunicação sem fio
            \item Microcontrolador
        \end{enumerate}


%-----------------------------------------------------------------
%               INTRODUÇÃO E JUSTIFICATIVA
%-----------------------------------------------------------------
\section{Introdução e Justificativa}

O texto de introdução deve conter três tipos de informações: apresentação do problema, estado da arte e justificativa do projeto.  Sendo que o autor deve elaborar
o texto com fluência lógica e sem redundância de informações.

A apresentação ou formulação do problema deve deixar, de forma bem clara, qual será o objeto de estudo do projeto.

O estado da arte serve para embasar tanto a formulação do problema como sua justificativa. É preciso situar historicamente a evolução do tema,
quais as abordagens já investigadas, qual o estágio atual do conhecimento sobre o assunto ou quais as tendências que se apresentam.

A justificativa do projeto deve indicar por que o projeto deve ser feito. Descreva os fatores de motivação que o levaram a abordar e trabalhar no assunto.

%-----------------------------------------------------------------
%                       OBJETIVO
%-----------------------------------------------------------------
\section{Objetivos}



Os objetivos devem ser claros, sucintos e diretos. Deve ficar bem evidente qual a pergunta ou questionamento para o qual se busca uma resposta através desta pesquisa.

Os objetivos são divididos em dois tipos: Objetivo Geral e Objetivos Específicos.
%-------------------------------------------------------------------------------

\subsection{Objetivo Geral}

Como pode ser notado, o título está no singular. Portanto, deve ser apresentado apenas 1 (um) objetivo geral. Aqui deve constar um parágrafo descrevendo esse objetivo.
%-------------------------------------------------------------------------------

\subsection{Objetivos Específicos}

O título está no plural. Portanto, espera-se encontrar mais de um objetivo específico neste local. Não confunda objetivo específico com metodologia.
Os objetivos específicos são o desdobramento do objetivo geral. Pode-se começar esse tópico conforme paragrafo a seguir.

{\bf Dentre os principais objetivos específicos destacam-se: }

Cada objetivo específico será colocado em forma de item e terá uma frase curta, mas que deixe claro qual o objetivo.
A somatória dos objetivos específicos formará o objetivo geral.
%-------------------------------------------------------------------------------


%-----------------------------------------------------------------
%           PLANO DE TRABALHO E CRONOGRAMA DE EXECUÇÃO
%-----------------------------------------------------------------
\section{Plano de Trabalho e Cronograma de Execução}

    Descrição das atividades planejadas para o desenvolvimento do trabalho e apresentação do cronograma de
execução destas atividades, respeitando as datas estabelecidas no cronograma da disciplina.


    Segue exemplo de como criar itens para descrever as atividades do projeto:

    \begin{enumerate}

        \item Atividades X: descrever o que será feito durante esta atividade;   \label{a1}

        \item Atividades Y: descrever o que será feito durante esta atividade;   \label{a2}

        \item Atividades Z: descrever o que será feito durante esta atividade;   \label{a3}

    \end{enumerate}

    Na Tabela \ref{tabela:cronograma1} é apresentado um exemplo de cronograma.

\begin{table}[ht]
    \scriptsize
    \centering
    \begin{tabular}{|l|c|c|c|c|c|c|c|c|c|}
        \hline &  \multicolumn{9}{|c|}
        {\textbf{Período}} \\ \cline{2-10}
        \textbf{Atividades}     &Mar      &Abr      &Mai      &Jun      &Jul      &Ago      &Set      &Out   &Nov \\ \hline \hline
        \ref{a1} - Atividade X  &$\bullet$&$\bullet$&         &         &         &         &         &      &    \\ \hline
        \ref{a2} - Atividade Y  &         &$\bullet$&$\bullet$&         &         &         &         &      &    \\ \hline
        \ref{a3} - Atividade Z  &         &         &$\bullet$&$\bullet$&$\bullet$&$\bullet$&         &      &   \\ \hline
    \end{tabular}
     \caption{Modelo 1 de Cronograma das Atividades}
    \label{tabela:cronograma1}
\end{table}

 Na Tabela \ref{tabela:cronograma2} é apresentado um outro exemplo de cronograma.

 \begin{table}[ht]
    \scriptsize
    \centering
    \begin{tabular}{|l|c|c|c|c|} \hline
        \textbf{Atividades}     & Inicio  &Término      &Duração     &Resultado  \\ \hline \hline
        \ref{a1} - Atividade X  & 01/03   &31/03        & 1 mês      & Proposta de Trabalho \\ \hline
        \ref{a2} - Atividade Y  & 01/04   &30/05        & 2 meses    & Revisão bibligráfica  \\ \hline
        \ref{a3} - Atividade Z  & 01/06   &31/10        & 5 meses    & Desenvolvimento \\ \hline
    \end{tabular}
    \caption{Modelo 2 de Cronograma das Atividades}
    \label{tabela:cronograma2}
\end{table}

%-----------------------------------------------------------------
%                   MATERIAL E MÉTODO
%-----------------------------------------------------------------

\section{Material e Método}

Texto que descreve os materiais/recursos que serão utilizados durante o desenvolvimento do trabalho
e também os métodos que serão utilizados para o desenvolvimento das atividades apresentadas na seção
anterior.

Os próximos  parágrafos exemplificam como fazer referências durante o texto.

Diversos métodos têm sido propostos para a análise de dados faltantes em conjuntos de dados \cite{monard2003}.

Nesse parágrafo é apresentado um exemplo de como fazer citação faz parte do texto. Segundo \citeonline{ib2008}, a estrutura ...

A diferença entre as diferentes formas de citação são exemplificadas no próximo paragráfo.

As maneiras mais comuns de citações são a indireta e a direta. Na citação indireta, o texto é criado com base na obra de autor consultado,
no qual se reproduz o conteúdo e as idéias do documento original. Exemplo, utilizando Sobrenome do Autor (data), quando o nome do autor
faz parte do texto: Segundo Souza (1999), a importância do tema [...]. Exemplo, utilizando (SOBRENOME DO AUTOR, ano) quando é citada a
síntese de uma informação: A Revolução Industrial modificou definitivamente o cenário urbano (SOUZA, 2001). Na citação direta há a
reprodução exata do texto citado entre aspas, como, por exemplo: A justificativa deste comportamento "é resultado da integração entre
parasita e hospedeiro, após a conclusão da fase de migração" (SOUZA, 1987).

%-----------------------------------------------------------------
%               CRITÉRIOS DE AVALIAÇÃO
%-----------------------------------------------------------------
\section{Critérios de Avaliação}

Texto que descreve a forma como os resultados que se espera obter do trabalho serão avaliados, analisados, medidos.

%-----------------------------------------------------------------
%                       REFERÊNCIAS
%-----------------------------------------------------------------

\section{Referências}

Referencias bibliográficas que foram utilizadas para desenvolver a proposta de TCC.
    \vspace{-4.3em}
    \renewcommand\refname{}

    \bibliography{referencias}

%-----------------------------------------------------------------
%                   SÍNTESE BIBLIOGRÁFICA
%-----------------------------------------------------------------
\section{Síntese Bibliográfica}

Referencias bibliográficas que se pretende utilizar para o desenvolvimento do trabalho.

\bibitem[Atmel Corporation 2009]{avrdatasheet}
\abntrefinfo{Atmel Corporation}{ATMEL CORPORATION}{2009}
{ATMEL CORPORATION. \emph{8-bit AVR Microcontroller with 4/8/16/32K Bytes
  In-System Programmable Flash}.
San Jose, CA, 2009.}

\bibitem[Dargie e Poellabauer 2010]{dargie_poellabauer2010}
\abntrefinfo{Dargie e Poellabauer}{DARGIE; POELLABAUER}{2010}
{DARGIE, W.; POELLABAUER, C. \emph{Fundamentals of wireless sensor networks}.
  Chichester, West Sussex, U.K.: Wiley, 2010.}

\bibitem[Gadre 2001]{gadre2001}
\abntrefinfo{Gadre}{GADRE}{2001}
{GADRE, D.~V. \emph{Programming and customizing the AVR microcontroller}. New
  York: McGraw-Hill, 2001.}

\bibitem[Karl e Willig 2005]{karl_willig2005}
\abntrefinfo{Karl e Willig}{KARL; WILLIG}{2005}
{KARL, H.; WILLIG, A. \emph{Protocols and architectures for wireless sensor
  networks}. Hoboken, NJ: Wiley, 2005.}

\bibitem[L\'opez e Zhou 2008]{lopez_zhou2008}
\abntrefinfo{L\'opez e Zhou}{L\'OPEZ; ZHOU}{2008}
{L\'OPEZ, J.; ZHOU, J. \emph{Wireless sensor network security}. Amsterdam: IOS
  Press, 2008.}

\bibitem[Nordic Semiconductor 2008]{nrfdatasheet}
\abntrefinfo{Nordic Semiconductor}{NORDIC SEMICONDUCTOR}{2008}
{NORDIC SEMICONDUCTOR. \emph{nRF24L01+: Single Chip 2.4GHz Transceiver -
  Product Specification v1.0}.
Trondheim, 2008.}

\bibitem[Trevennor 2012]{trevennor2012}
\abntrefinfo{Trevennor}{TREVENNOR}{2012}
{TREVENNOR, A. \emph{Practical AVR microcontrollers}. Berkeley, CA: Apress,
  2012.}

\bibitem[Williams 2014]{williams2014}
\abntrefinfo{Williams}{WILLIAMS}{2014}
{WILLIAMS, E. \emph{Make: AVR Programming}. Sebastopol, Calif.: Maker Media,
  2014.}

\bibitem[Aldrich 2003]{aldrich2003}
\abntrefinfo{Aldrich}{ALDRICH}{2003}
{ALDRICH, F.~K. Smart homes: Past, present and future. In:  HARPER, R. (Ed.).
  \emph{Inside the Smart Home}. London: Springer, 2003.}

\bibitem[Buratti 2011]{buratti2011}
\abntrefinfo{Buratti}{BURATTI}{2011}
{BURATTI, C. \emph{Sensor networks with IEEE 802.15.4 systems}. Berlin:
  Springer, 2011.}

\bibitem[Hagen 2002]{hagen2002}
\abntrefinfo{Hagen}{HAGEN}{2002}
{HAGEN, S. \emph{IPV6 essentials}. Beijing: O'Reilly, 2002.}

\bibitem[Harper 2003]{harper2003}
\abntrefinfo{Harper}{HARPER}{2003}
{HARPER, R. Inside the smart home: Ideas, possibilities and methods. In:
  HARPER, R. (Ed.). \emph{Inside the Smart Home}. London: Springer, 2003.}

\bibitem[Kuorilehto et al. 2007]{kuorilehto2007}
\abntrefinfo{Kuorilehto et al.}{KUORILEHTO et al.}{2007}
{KUORILEHTO, M. et al. \emph{Ultra-low energy wireless sensor networks in
  practice}. Chichester, England: John Wiley & Sons, 2007.}

\bibitem[Kyas 2013]{kyas2013}
\abntrefinfo{Kyas}{KYAS}{2013}
{KYAS, O. \emph{How To Smart Home}. Wyk, Germany: Key Concept Prees e.K.,
  2013.}

\bibitem[Riley 2012]{riley2012}
\abntrefinfo{Riley}{RILEY}{2012}
{RILEY, M. \emph{Programming your home}. Dallas, Tex.: Pragmatic Bookshelf,
  2012.}

\bibitem[Shelby e Bormann 2009]{shelby_bormann2009}
\abntrefinfo{Shelby e Bormann}{SHELBY; BORMANN}{2009}
{SHELBY, Z.; BORMANN, C. \emph{6LoWPAN: The Wireless Embedded Internet}.
  Chichester, U.K.: J. Wiley, 2009.}

\bibitem[Sohraby, Minoli e Znati 2007]{sohraby_minoli_znati2007}
\abntrefinfo{Sohraby, Minoli e Znati}{SOHRABY; MINOLI; ZNATI}{2007}
{SOHRABY, K.; MINOLI, D.; ZNATI, T.~F. \emph{Wireless sensor networks}.
  Hoboken, N.J.: Wiley-Interscience, 2007.}

\bibitem[Texas Instruments 2015]{ccdatasheet}
\abntrefinfo{Texas Instruments}{TEXAS INSTRUMENTS}{2015}
{TEXAS INSTRUMENTS. \emph{CC2500 - Low-Cost Low-Power 2.4 GHz RF Transceiver}.
Dallas, Texas, 2015.}




\end{document}


