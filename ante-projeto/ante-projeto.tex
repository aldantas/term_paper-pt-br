%-----------------------------------------------------------------
% UNIOESTE - Ciência da Computação
% 4o. ano - Trabalho de Conclusão de Curso
% Profas. Teresinha Arnauts Hachisuca e Izaura
%-----------------------------------------------------------------
%DECLARAÇÃO DO TIPO DE DOCUMENTO, TAMANHO DA FOLHA E FONTE
 %-----------------------------------------------------------------
\documentclass[
    % -- opções da classe memoir --
    12pt,               % tamanho da fonte
%   twoside,            % para impressão em verso e anverso. Oposto a oneside
    a4paper,            % tamanho do papel.
    % -- opções da classe abntex2 --
    %chapter=TITLE,     % títulos de capítulos convertidos em letras maiúsculas
    %section=TITLE,     % títulos de seções convertidos em letras maiúsculas
    %subsection=TITLE,  % títulos de subseções convertidos em letras maiúsculas
    %subsubsection=TITLE,% títulos de subsubseções convertidos em letras maiúsculas
    % -- opções do pacote babel --
    english,            % idioma adicional para hifenização
    brazil,             % o último idioma é o principal do documento
    ]{article}

%-----------------------------------------------------------------
%DEFINIÇÕES DOS PACOTES UTILIZADOS
%-----------------------------------------------------------------
\usepackage{cmap}               % Mapear caracteres especiais no PDF
\usepackage{lmodern}            % Usa a fonte Latin Modern
\usepackage[T1]{fontenc}        % Selecao de codigos de fonte.
\usepackage[utf8]{inputenc}     % Codificacao do documento (conversão automática dos acentos)
\usepackage{lastpage}           % Usado pela Ficha catalográfica
%\usepackage{indentfirst}       % Indenta o primeiro parágrafo de cada seção.
\usepackage{color}              % Controle das cores
\usepackage{graphicx}           % Inclusão de gráficos

\usepackage[brazil]{babel}      % Permite traduzir termos do LateX para português Brasil.
\usepackage{hyperref}           % Permite ativar hyperlinks
\usepackage{algorithm}          % pacote que de suporte a representação de algoritmos.

\usepackage{algorithmic}
\usepackage[pdftex]{geometry}
\usepackage{multirow}
\usepackage{siunitx}

% ---
% Pacotes de citações
% ---
\usepackage[brazilian,hyperpageref]{backref}     % Paginas com as citações na bibl
\usepackage[alf]{abntex2cite}   % Citações padrão ABNT
%\citebrackets[]

% ---
% CONFIGURAÇÕES DE PACOTES
% ---

% ---
% Configurações do pacote backref
% Usado sem a opção hyperpageref de backref
\renewcommand{\backrefpagesname}{Citado na(s) página(s):~}

% Texto padrão antes do número das páginas
\renewcommand{\backref}{}

% Define os textos da citação
\renewcommand*{\backrefalt}[4]{
    \ifcase #1 %
        Nenhuma citação no texto.%
    \or
        Citado na página #2.%
    \else
        Citado #1 vezes nas páginas #2.%
    \fi}%
% ---

%-----------------------------------------------------------------
%               INICIO DO DOCUMENTO - CAPA
%-----------------------------------------------------------------
\begin{document}


\begin{center}

    \textsc{
        \large
            \\universidade estadual do oeste do paraná
            \\unioeste - campus de foz do iguaçu
            \\centro de engenharias e ciências exatas
            \\curso de ciência da computação
            \\[1 cm]tcc - trabalho de conclusão de curso
    }
    \\
    [4 cm]
    \large Proposta de Trabalho de Conclusão de Curso
    \\
    \textbf{
        \textsc{Desenvolvimento de uma Rede de Sensores sem Fio de baixo custo}}
    }
    \\[5 cm]Augusto Lopez Dantas
    \\Orientador: Jorge Habib Hanna El Khouri
    \\[2 cm]Foz do Iguaçu, 04 de maio de 2015

\end{center}

\thispagestyle{empty}

%-----------------------------------------------------------------
%               IDENTIFICAÇÃO DO PROJETO
%-----------------------------------------------------------------
\section{Identificação}

    \subsection{Área e Linha de Pesquisa}
        \noindent Grande Área: Ciências Exatas e da Terra
        \\Código: 10000003
        \\[1 cm]Linha de Pesquisa: Ciência da Computação
        \\Código: 10300007
        \\[1 cm]Especialidade: \{nome da especialidade\}
        \\Código: \{código da especialidade\}

    \subsection{Palavras-chave}

        \begin{enumerate}
            \item Domótica
			\item Rede de Sensores Sem Fio
			\item Programação Embarcada
        \end{enumerate}


%-----------------------------------------------------------------
%               INTRODUÇÃO E JUSTIFICATIVA
%-----------------------------------------------------------------
\section{Introdução e Justificativa}
Ao longo dos anos, a computação e suas aplicações se tornaram cada vez mais presentes no nosso cotidiano, sendo apenas uma questão
de tempo para que ela fosse ampliada com o intuito de melhorar o controle e eficiência dos ambientes domésticos e empresariais.

Embora já na década de 60 casas eletricamente sofisticadas eram construídas, foi durante a primeira metáde dá década de 1980 que o
termo  ``casa inteligente'' passou a ser utilizado. Pois o que determina uma casa inteligente não é o quão bem é construída ou
quão efetiva ela é em relação ao espaço, mas sim as tecnologias interativas que ela contém. \cite{harper2003}

Para \citeonline{aldrich2003}, uma casa inteligente pode ser definida como uma residência equipada de tecnologias computacionais
que antecipam e respondem às necessidades dos ocupantes, promovendo conforto, conveniência, segurança e entretenimento através do
gerenciamento da tecnologia interna e conexões com o mundo a fora.

Este conceito relativamente recente ficou conhecido como domótica, cujo termo origina-se da junção das palavras \textit{domus},
latim para "casa", e robótica, que remete à automação. Sendo assim, domótica pode ser traduzida como automação residencial, porém,
grande parte de seus conceitos também pode ser aplicada a diversos ambientes como escritórios, indústrias, entre outros.

Segundo \citeonline{riley2012}, domótica é um produto ou serviço que proporciona algum nível de ação ou mensagem para o ambiente
domiciliar, um evento que foi gerado sem a intervenção direta do morador. Um despertador ou um alarme de incêndio são exemplos
disso, porém, esses dispositivos autônomos não necessariamente possuem um mecanismo de comunicação entre eles, limitando o nível
da automação e inteligência da solução.

Sob uma perspectiva técnica, automação residencial consiste em cinco princípios: dispostivos sob controle, que são todos os
eletrodomésticos e eletrônicos de consumo que estão conectados e controlados pelo mesmo sistema de automatização; sensores e
atuadores, que agem como os olhos e mãos da rede residencial medindo e controlando o ambiente; dispositivos de controle remoto,
como um \textit{smartphone}, que possibilitam a interação do usuário com a aplicação; controlador, um sistema computacional que
coleta informações através dos sensores e recebe comandos através do dispositivo de controle remoto; e redes de controle, que
fornecem a comunicação entre as tecnologias envolvidas. \cite{kyas2013}

Em relação à rede de controle, essa pode existir em três formatos: sem fio, com fio ou utilizando cabos energia elétrica. As três
tecnologias têm melhorado significamente em termos de velocidade, confiabilidade e interoperabilidade através da padronização nos
últimos dez anos. \cite{kyas2013}

Contudo, ainda hoje não há um protocolo de comunicação que seja inteiramente adotado, parcialmente devido ao fato de requerer que
os fabricantes concordem em criar eletrônicos com as mesmas interfaces e protocolos designados em seus produtos. Porém, as
tentativas em padronizar a comunicação na domótica existem desde quase seu início. Uma das primeiras tentativas foi o X10, que
utilizava a rede elétrica existente para transmitir mensagens através de pulsos codificados, mas devido a diversos problemas, como
degradação de sinal e dificuldade para verificação de pacotes, ele acabou não ganhando muita aceitação. \cite{riley2012}

A expectativa é que todos os dispositivos eletroeletrônicos possuam conexão à internet, conceito que vem sendo chamado de
Internet das Coisas e que prevê a adressabilidade única para todos os nós conectados na rede mundial através da implantação do
IPv6. Este último é uma evolução do protocolo de rede IPv4 e surgiu em 1994 com a intenção de resolver a limitação de espaço de
endereço prevista e fornecer funcionalidades adicionais. \cite{hagen2002}

Entretanto, a adoção do IPv6 de forma unânime ainda não aconteceu mesmo depois de duas décadas desde sua criação, além disso, sua
utilização completa em domótica levanta a questão quanto à necessidade em conecetar dispositivos simples como sensores e atuadores
à \textit{internet}, levando em consideração custo de implementação e vulnerabilidade. Alternativamente, existe a possibilidade em
desenvolver uma rede específica para esses transdutores, permitindo implementar a monitoria de um ambiente inteligente de maneira
simples e eficáz.

Essa rede específica, conhecida como rede de sensores [e atuadores], é uma infraestrutura composta por dispositivos de medição e
atuação, de computação e de comunicação que possibilitam instrumentar, observar e reagir à eventos e fenômenos em um determinado
ambiente. \cite{sohraby_minoli_znati2007}

Os quantitativos físicos que podem ser medidos são diversos, como temperatura, umidade, som, luz, radiação, etc. Como
consequência, são várias as possibilidades de aplicação. Algumas das mais comuns são automação residencial e comercial,
monitoramento ambiental e industrial, controle de tráfego, militarismo e assistência médica. A complexidade da rede deve ser
proporcial à sua aplicação e vários aspectos devem ser levados em consideração, como por exemplo o meio de comunicação optado.
\cite{kuorilehto2007}

A utilização de fios para realizar tal tarefa é normalmente descartada devido à diversos fatores como custo, manutenção,
localização e mobilidade, portanto, uma comunicação sem fio, na maioria dos cenários, é um requerimento inevitável. Com isso,
surgiu o termo Rede de Sensores Sem Fio (RSSF), que, apesar do nome, frequentemente também inclui atuadores.
\cite{karl_willig2005}

No caso da aplicação deste conceito em domótica, é preferível que a implementação da RSSF seja de forma simples, a fim de obter
uma solução com baixo custo e baixo consumo de eletrecidade, sendo assim, é necessário escolher a tecnologia adequada para tal
feito. O protocolo Wi-Fi, por exemplo, foi taxado como muito complexo e com suporte a largura de banda maior que o necessário, já
sistemas infravermelho requerem uma linha de visão, o que nem sempre é possível neste caso. A tecnologia Bluetooth, por sua vez,
aparentou ser promissora no início mas logo foi julgada como cara e complexa. Isso acabou abrindo as portas para a criação do
padrão IEEE 802.15.4 no ano de 2003.  \cite{sohraby_minoli_znati2007}

A tecnologia IEEE 802.15.4 é um sistema de comunicação por rádio frequência de baixo alcance projetado para ter baixa
complexidade, baixo custo, baixo consumo de energia e baixa taxa de transmissão de dados. Esse padrão implementa as camadas física
e de enlace do modelo ISO/OSI e é amplamente utilizado na construção de RSSFs. Além disso, serve como base para outros protocolos
que implementam camadas superiores, como por exemplo o protocolo ZigBee, que fornece mecanismos para entrada e saída em uma rede,
segurança de pacotes, roteamento, descoberta de caminho (para casos de rede malha), entre outros recursos. \cite{buratti2011}

Outro protocolo que se baseia no padrão IEEE 802.15.4 é o 6LoWPAN(\textit{IPv6 over Low power Wireless Personal Area Networks}),
que possibilita o uso eficiente do IPv6 sobre redes sem fio de baixa taxa e baixo consumo de energia em dispositivos embarcados
simples através de uma camada de adaptação e da otimização dos protocolos relacionados. \cite{shelby_bormann2009}

Porém, os módulos que utilizam esses protocolos sofisticados possuem um preço relativamente maior em relação aos módulos com
funcionamento mais básico e acabam inflando o custo total da RSSF de acordo com o número de nós da mesma. Portanto, a fim de se
obter uma solução barata é necessário utilizar um transceptor simples e uma unidade de controle externa de baixo custo para
implementar demais funcionalidades não realizadas pelo módulo.

Uma possibilidade é utilizar o transceptor CC2500 da empresa \textit{Texas Instruments}, que implementa o padrão IEEE 802.15.4 e
possui uma taxa máxima de transmissão aérea de 500 Kbps, consumindo \SI{17}{\milli \ampere} durante recepção, \SI{21.2}{\milli
\ampere} durante transmissão e \SI{1.5}{\milli \ampere} em modo de espera. Devido ao baixo consumo elétrico e ótimo
custo-benefício, esse dispositivo é amplamente utilizado. \cite{ccdatasheet}

Outro transceptor de rádio frequência com característica simplista bastante difundido é o nRF24L01+ da empresa \textit{Nordic
Semiconductor}. Embora possui a desvantagem de não seguir o padrão aberto da IEEE, esse dispositivo apresenta diversas vantagens
em relação aos demais módulos dessa categoria. Uma das principais vantagens é sua taxa máxima de transmissão aérea de 2 Mpbs
(quatro vezes mais que o CC2500) e que, ao mesmo tempo, consome menos energia elétrica que os demais, sendo \SI{13.5}{\milli
\ampere} durante recepção, \SI{11.3}{\milli \ampere} durante transmissão e \SI{26}{\micro \ampere} em modo de espera.
\cite{nrfdatasheet}

Além disso, o nRF24L01+ oferece serviços como reconhecimento e retransmissão de pacotes automáticos, reduzindo o número de
comunicação com a unidade microtronladora tal como o processamento utilizado pela mesma. Dessa forma, além de reduzir ainda mais
o consumo elétrico necessário para a solução, torna possível uma implementação eficiente utilizando microcontroladores simples e
baratos. \cite{nrfdatasheet}

Portanto, sabe-se que é possível desenvolver uma Rede de Sensores Sem Fio para aplicação em automação de residências e escritórios
de maneira simples e eficáz utilizando apenas tecnologias já existentes a acessíveis.


%-----------------------------------------------------------------
%                       OBJETIVO
%-----------------------------------------------------------------
\section{Objetivos}



Os objetivos devem ser claros, sucintos e diretos. Deve ficar bem evidente qual a pergunta ou questionamento para o qual se busca uma resposta através desta pesquisa.

Os objetivos são divididos em dois tipos: Objetivo Geral e Objetivos Específicos.
%-------------------------------------------------------------------------------

\subsection{Objetivo Geral}

Como pode ser notado, o título está no singular. Portanto, deve ser apresentado apenas 1 (um) objetivo geral. Aqui deve constar um parágrafo descrevendo esse objetivo.
%-------------------------------------------------------------------------------

\subsection{Objetivos Específicos}

O título está no plural. Portanto, espera-se encontrar mais de um objetivo específico neste local. Não confunda objetivo específico com metodologia.
Os objetivos específicos são o desdobramento do objetivo geral. Pode-se começar esse tópico conforme paragrafo a seguir.

{\bf Dentre os principais objetivos específicos destacam-se: }

Cada objetivo específico será colocado em forma de item e terá uma frase curta, mas que deixe claro qual o objetivo.
A somatória dos objetivos específicos formará o objetivo geral.
%-------------------------------------------------------------------------------


%-----------------------------------------------------------------
%           PLANO DE TRABALHO E CRONOGRAMA DE EXECUÇÃO
%-----------------------------------------------------------------
\section{Plano de Trabalho e Cronograma de Execução}

    Descrição das atividades planejadas para o desenvolvimento do trabalho e apresentação do cronograma de
execução destas atividades, respeitando as datas estabelecidas no cronograma da disciplina.


    Segue exemplo de como criar itens para descrever as atividades do projeto:

    \begin{enumerate}

        \item Atividades X: descrever o que será feito durante esta atividade;   \label{a1}

        \item Atividades Y: descrever o que será feito durante esta atividade;   \label{a2}

        \item Atividades Z: descrever o que será feito durante esta atividade;   \label{a3}

    \end{enumerate}

    Na Tabela \ref{tabela:cronograma1} é apresentado um exemplo de cronograma.

% \begin{table}[ht]
%     \scriptsize
%     \centering
%     \begin{tabular}{|l|c|c|c|c|c|c|c|c|c|}
%         \hline &  \multicolumn{9}{|c|}
%         {\textbf{Período}} \\ \cline{2-10}
%         \textbf{Atividades}     &Mar      &Abr      &Mai      &Jun      &Jul      &Ago      &Set      &Out   &Nov \\ \hline \hline
%         \ref{a1} - Atividade X  &$\bullet$&$\bullet$&         &         &         &         &         &      &    \\ \hline
%         \ref{a2} - Atividade Y  &         &$\bullet$&$\bullet$&         &         &         &         &      &    \\ \hline
%         \ref{a3} - Atividade Z  &         &         &$\bullet$&$\bullet$&$\bullet$&$\bullet$&         &      &   \\ \hline
%     \end{tabular}
%      \caption{Modelo 1 de Cronograma das Atividades}
%     \label{tabela:cronograma1}
% \end{table}

 Na Tabela \ref{tabela:cronograma2} é apresentado um outro exemplo de cronograma.

%  \begin{table}[ht]
%     \scriptsize
%     \centering
%     \begin{tabular}{|l|c|c|c|c|} \hline
%         \textbf{Atividades}     & Inicio  &Término      &Duração     &Resultado  \\ \hline \hline
%         \ref{a1} - Atividade X  & 01/03   &31/03        & 1 mês      & Proposta de Trabalho \\ \hline
%         \ref{a2} - Atividade Y  & 01/04   &30/05        & 2 meses    & Revisão bibligráfica  \\ \hline
%         \ref{a3} - Atividade Z  & 01/06   &31/10        & 5 meses    & Desenvolvimento \\ \hline
%     \end{tabular}
%     \caption{Modelo 2 de Cronograma das Atividades}
%     \label{tabela:cronograma2}
% \end{table}

%-----------------------------------------------------------------
%                   MATERIAL E MÉTODO
%-----------------------------------------------------------------

\section{Material e Método}

Texto que descreve os materiais/recursos que serão utilizados durante o desenvolvimento do trabalho
e também os métodos que serão utilizados para o desenvolvimento das atividades apresentadas na seção
anterior.
%-----------------------------------------------------------------
%               CRITÉRIOS DE AVALIAÇÃO
%-----------------------------------------------------------------
\section{Critérios de Avaliação}

Texto que descreve a forma como os resultados que se espera obter do trabalho serão avaliados, analisados, medidos.

%-----------------------------------------------------------------
%                       REFERÊNCIAS
%-----------------------------------------------------------------

\section{Referências}

    \vspace{-4.3em}
    \renewcommand\refname{}

    \bibliography{bibliografia}

%-----------------------------------------------------------------
%                   SÍNTESE BIBLIOGRÁFICA
%-----------------------------------------------------------------
\section{Síntese Bibliográfica}

Referencias bibliográficas que se pretende utilizar para o desenvolvimento do trabalho.

% \bibitem[Atmel Corporation 2009]{avrdatasheet}
\abntrefinfo{Atmel Corporation}{ATMEL CORPORATION}{2009}
{ATMEL CORPORATION. \emph{8-bit AVR Microcontroller with 4/8/16/32K Bytes
  In-System Programmable Flash}.
San Jose, CA, 2009.}

\bibitem[Dargie e Poellabauer 2010]{dargie_poellabauer2010}
\abntrefinfo{Dargie e Poellabauer}{DARGIE; POELLABAUER}{2010}
{DARGIE, W.; POELLABAUER, C. \emph{Fundamentals of wireless sensor networks}.
  Chichester, West Sussex, U.K.: Wiley, 2010.}

\bibitem[Gadre 2001]{gadre2001}
\abntrefinfo{Gadre}{GADRE}{2001}
{GADRE, D.~V. \emph{Programming and customizing the AVR microcontroller}. New
  York: McGraw-Hill, 2001.}

\bibitem[Karl e Willig 2005]{karl_willig2005}
\abntrefinfo{Karl e Willig}{KARL; WILLIG}{2005}
{KARL, H.; WILLIG, A. \emph{Protocols and architectures for wireless sensor
  networks}. Hoboken, NJ: Wiley, 2005.}

\bibitem[L\'opez e Zhou 2008]{lopez_zhou2008}
\abntrefinfo{L\'opez e Zhou}{L\'OPEZ; ZHOU}{2008}
{L\'OPEZ, J.; ZHOU, J. \emph{Wireless sensor network security}. Amsterdam: IOS
  Press, 2008.}

\bibitem[Nordic Semiconductor 2008]{nrfdatasheet}
\abntrefinfo{Nordic Semiconductor}{NORDIC SEMICONDUCTOR}{2008}
{NORDIC SEMICONDUCTOR. \emph{nRF24L01+: Single Chip 2.4GHz Transceiver -
  Product Specification v1.0}.
Trondheim, 2008.}

\bibitem[Trevennor 2012]{trevennor2012}
\abntrefinfo{Trevennor}{TREVENNOR}{2012}
{TREVENNOR, A. \emph{Practical AVR microcontrollers}. Berkeley, CA: Apress,
  2012.}

\bibitem[Williams 2014]{williams2014}
\abntrefinfo{Williams}{WILLIAMS}{2014}
{WILLIAMS, E. \emph{Make: AVR Programming}. Sebastopol, Calif.: Maker Media,
  2014.}

\bibitem[Aldrich 2003]{aldrich2003}
\abntrefinfo{Aldrich}{ALDRICH}{2003}
{ALDRICH, F.~K. Smart homes: Past, present and future. In:  HARPER, R. (Ed.).
  \emph{Inside the Smart Home}. London: Springer, 2003.}

\bibitem[Buratti 2011]{buratti2011}
\abntrefinfo{Buratti}{BURATTI}{2011}
{BURATTI, C. \emph{Sensor networks with IEEE 802.15.4 systems}. Berlin:
  Springer, 2011.}

\bibitem[Hagen 2002]{hagen2002}
\abntrefinfo{Hagen}{HAGEN}{2002}
{HAGEN, S. \emph{IPV6 essentials}. Beijing: O'Reilly, 2002.}

\bibitem[Harper 2003]{harper2003}
\abntrefinfo{Harper}{HARPER}{2003}
{HARPER, R. Inside the smart home: Ideas, possibilities and methods. In:
  HARPER, R. (Ed.). \emph{Inside the Smart Home}. London: Springer, 2003.}

\bibitem[Kuorilehto et al. 2007]{kuorilehto2007}
\abntrefinfo{Kuorilehto et al.}{KUORILEHTO et al.}{2007}
{KUORILEHTO, M. et al. \emph{Ultra-low energy wireless sensor networks in
  practice}. Chichester, England: John Wiley & Sons, 2007.}

\bibitem[Kyas 2013]{kyas2013}
\abntrefinfo{Kyas}{KYAS}{2013}
{KYAS, O. \emph{How To Smart Home}. Wyk, Germany: Key Concept Prees e.K.,
  2013.}

\bibitem[Riley 2012]{riley2012}
\abntrefinfo{Riley}{RILEY}{2012}
{RILEY, M. \emph{Programming your home}. Dallas, Tex.: Pragmatic Bookshelf,
  2012.}

\bibitem[Shelby e Bormann 2009]{shelby_bormann2009}
\abntrefinfo{Shelby e Bormann}{SHELBY; BORMANN}{2009}
{SHELBY, Z.; BORMANN, C. \emph{6LoWPAN: The Wireless Embedded Internet}.
  Chichester, U.K.: J. Wiley, 2009.}

\bibitem[Sohraby, Minoli e Znati 2007]{sohraby_minoli_znati2007}
\abntrefinfo{Sohraby, Minoli e Znati}{SOHRABY; MINOLI; ZNATI}{2007}
{SOHRABY, K.; MINOLI, D.; ZNATI, T.~F. \emph{Wireless sensor networks}.
  Hoboken, N.J.: Wiley-Interscience, 2007.}

\bibitem[Texas Instruments 2015]{ccdatasheet}
\abntrefinfo{Texas Instruments}{TEXAS INSTRUMENTS}{2015}
{TEXAS INSTRUMENTS. \emph{CC2500 - Low-Cost Low-Power 2.4 GHz RF Transceiver}.
Dallas, Texas, 2015.}




\end{document}


