\begin{resumo}

A integração de sensores em ambientes domiciliares e em escritórios em conjunto com a entrega eficiente das
informações observadas resulta em uma aprimoração na comodidade e segurança. Dessa forma, tem-se as chamadas
redes de sensores sem fio (RSSF), que, apesar do nome, também incluem atuadores e que podem atuar como um dos
componontes fundamentais em um sistema de automação residencial. Este trabalho tem como objetivo ratificar a
possibilidade da construção de uma RSSF barata e econômica com capacidade para atuar em ambientes não
crı́ticos. Através de uma análise dos dispositivos existentes que atuam nesse âmbito, selecionou-se aqueles que
eram mais compatı́veis com o resultado desejado. A partir de então, inicou-se o processo de estudo e
desenvolvimento da solução, obtendo como resultado uma implementação de RSSF para ambientes não críticos com uma
melhor eficiência energética e um custo monetário significativamente menor do que as soluções normalmente
encontradas no mercado. Essa abordagem possibilita com que mais pessoas tenham acesso a esse tipo de
tecnologia, permitindo-as usufruir das vantagens que um sistema de automação traz ao cotidiano.

\vspace{\onelineskip}
\noindent
\textbf{Palavras-chaves}: Rede de Sensores Sem Fio; Automação Residencial; Microcontrolador; Transceptor de
Rádiofrequência.
\end{resumo}
